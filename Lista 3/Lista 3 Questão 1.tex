\documentclass[12pt,a4paper]{article}
\usepackage[brazil]{babel}
\usepackage[utf8]{inputenc}
\usepackage[T1]{fontenc}
\usepackage{graphicx,color}
\usepackage{graphicx}
\title{Lista Métodos 2}
\author{José Augusto Câmara Filho 
 Matrícula: 358568}

\begin{document}
José Augusto Câmara Filho\\
Matrícula : 358568

1) Prove que o produto de duas matrizes ortogonais é uma matriz ortogonal.


Uma matriz quadrada é dita ortogonal quando a sua transposta coincide com a sua inversa. Isto é, uma matriz quadrada M é ortogonal se:

$$M^T=M^{-1}$$

Ou, alternativamente:

$$MM^T=I_n$$

Note que uma matriz é ortogonal se e somente se as colunas (ou linhas) são vetores ortonormais.
 
Prova: Sejam A e B matrizes ortogonais.

$$(AB)*(B^\top A^\top)= A(BB^\top) A^\top = A*I*A^\top =$$
$$=AA^\top$$
$$=I$$
Logo AB é ortogonal.

\end{document}